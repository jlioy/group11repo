\documentclass[letterpaper, onecolumn, draftclsnofoot, 10pt, compsoc]{IEEEtran}

\usepackage{graphicx}                                        
\usepackage{amssymb}                                         
\usepackage{amsmath}                                         
\usepackage{amsthm}                                          
\usepackage{alltt}                                           
\usepackage{float}
\usepackage{color}
\usepackage{url}
\usepackage{balance}
\usepackage{enumitem}
\usepackage{pstricks, pst-node}
\usepackage{setspace}

\usepackage{geometry}
\geometry{textheight=9.5in, textwidth=7in}

\usepackage{listings}

\definecolor{codegreen}{rgb}{0,0.6,0}
\definecolor{codegray}{rgb}{0.5,0.5,0.5}
\definecolor{codepurple}{rgb}{0.58,0,0.82}
\definecolor{backcolour}{rgb}{0.95,0.95,0.92}

\lstdefinestyle{mystyle}{
  backgroundcolor=\color{backcolour},  
  commentstyle=\color{codegreen},
  keywordstyle=\color{magenta},
  numberstyle=\tiny\color{codegray},
  stringstyle=\color{codepurple},
  basicstyle=\footnotesize,
  breakatwhitespace=false,         
  breaklines=true,                 
  captionpos=b,                    
  keepspaces=true,                 
  numbers=left,                    
  numbersep=5pt,                  
  showspaces=false,                
  showstringspaces=false,
  showtabs=false,                  
  tabsize=2
}

\lstset{style=mystyle}

\newcommand{\cred}[1]{{\color{red}#1}}
\newcommand{\cblue}[1]{{\color{blue}#1}}

\usepackage{hyperref}
\hypersetup{
    colorlinks=false,
    pdfborder={0 0 0},
}
\usepackage{geometry}

\def \name{Joshua Lioy}
\def \GroupNumber{11}
\def \GroupMemberOne{Joshua Lioy}
\def \GroupMemberTwo{Brian Wiltse}

\begin{document}
\begin{titlepage}
    \pagenumbering{gobble}
    \begin{singlespace}
    	\includegraphics[height=4cm]{coe_v_spot1}
        \hfill 
        \par\vspace{.2in}
        \centering
        \scshape{
            \huge Operating Systems 2 \par
            {\large\today}\par
            \vspace{.5in}
            \textbf{\Huge Project 1: Getting Acquainted}\par
            \vfill
            \vspace{5pt}
            {\large Prepared by }\par
            Group \GroupNumber\par
            \vspace{5pt}
            {\Large
                \GroupMemberOne\par
                \GroupMemberTwo\par
            }
            \vspace{20pt}
        }
        \begin{abstract}
        This document covers Group 11's findings for assignment 1. This includes the commands used for the initial kernel setup as well as a write-up of the concurrency solution for the Producer-Consumer problem.
        \end{abstract}     
    \end{singlespace}
\end{titlepage}

\newpage
\pagenumbering{arabic}
\tableofcontents

\newpage
\section{Log of Commands Used to Set Up and Run the Virtual Linux Kernel}

    \subsection{Setting Up the Environment}
    \begin{lstlisting}[language=bash]
    source /scratch/fall2017/11/common/environment-setup-i586-poky-linux 
    \end{lstlisting}
    
    \subsection{Checking Out v3.19.2}
    \begin{lstlisting}[language=bash]
    git checkout "v3.19.2"
    \end{lstlisting}
    
    \subsection{Running qemu the First Time}
    \begin{lstlisting}[language=bash]
    qemu-system-i386 -gdb tcp::5511 -S -nographic -kernel bzImage-qemux86.bin -drive file=core-image-lsb-sdk-qemux86.ext4,if=virtio -enable-kvm -net none -usb -localtime --no-reboot --append "root=/dev/vda rw console=ttyS0 debug"
    \end{lstlisting}
    
    \subsection{Running qemu}
    \begin{lstlisting}[language=bash, caption=Original Kernel]
    qemu-system-i386 -gdb tcp::5511 -S -nographic -kernel bzImage-qemux86.bin -drive file=core-image-lsb-sdk-qemux86.ext4,if=virtio -enable-kvm -net none -usb -localtime --no-reboot --append "root=/dev/vda rw console=ttyS0 debug"
    \end{lstlisting}
    \begin{lstlisting}[language=bash, caption=Our Kernel]
    qemu-system-i386 -gdb tcp::5511 -S -nographic -kernel /scratch/fall2017/11/group11repo/linux-yocto-3.19/arch/x86/boot/bzImage -drive file=/scratch/fall2017/11/common/core-image-lsb-sdk-qemux86.ext4,if=virtio -enable-kvm -net none -usb -localtime --no-reboot --append "root=/dev/vda rw console=ttyS0 debug"
    \end{lstlisting}
    
    \subsection{Connecting to the Kernel Remotely Through GDB}
    \begin{lstlisting}[language=bash]
    target remote :5511
    continue
    \end{lstlisting}
    
    \subsection{Shutting Down the Virtual Machine}
    \begin{lstlisting}[language=bash]
    shutdown -h now 
    \end{lstlisting}
    
    \subsection{Building a New Instance of the Kernel}
    \begin{lstlisting}[language=bash]
    make -j4 all
    \end{lstlisting}

\newpage
\section{An Explanation of the Flags in the qemu Command-line}
    \begin{singlespace}
        \begin{itemize}
            \item -gdb: This flag provides a gdb stub which allows connection from any machine on the network. In our case this is another connection through the OS2 server.
            \item -S: This flag enforces manual CPU start. In our case the CPU will start after entering continue into gdb.
            \item -nographic: This flag disables graphics.
            \item -kernel: This flag specifies the location of the kernel file. Note that multiboot is not supported.
            \item -drive: This flag specifies the location of the virtual drive file as well as options for things such as the media type, drive file location, interface etc.
            \item -enable-kvm: This flag forces the use of KVM Virtualization.
            \item -net: This flag configures the network parameters.
            \item -usb: This flag enables the use of usb devices.
            \item -localtime: This flag sets the virtual machine to use the time of the local machine that it is running on. In our case this is the OS2 server time.
            \item --no-reboot: This flag prevents rebooting of the virtual machine.
            \item --append: This flag appends a command line parameter.
        \end{itemize}
    \end{singlespace}

\section{Concurrency Solution Write-Up}
    \begin{singlespace}
    %Introduction Here
            We believe the point of this assignment is to  get us to work through a simulation of how event-driven scheduling works in the kernel.
            Synchronizing multiple threads and processes is essential for working with the kernel, and this is a minimal demonstration of how to correctly do so.   \par
            We began to address the problem by reading chapters 1-4 of The Little Book of Semaphores by Allan Downey \cite{DowneyBOS}. 
            Chapter 4, in particular, was helpful in determining how to synchronize the producers and consumers. 
            We used the design pattern outlined in 4.1.6.\par
            
            The solution uses 2 mutexes: one to block other producers and consumers from using the shared buffer when one was accessing it, and one to block our random generator function since we have multiple producers that might try to access the function simultaneously.
            \par
            
            To control how many items can be placed in the buffer, we used a semaphore to indicate how many empty slots were left in the buffer and a different semaphore to indicate how many slots were filled. 
            The former is initialized to the maximum buffer size and will block producers when the empty slots get down to zero; 
            the latter is initialized to 0 and will block the consumers when the full slots get down to zero.
            All mutexes and semaphores are global variables, as they are shared by producers and consumers.\par

            The consumer waits for the \texttt{sem\_full} semaphore to ensure that the full slots in the buffer has not reduced to zero.
            It also waits for the mutex on the buffer to ensure another consumer or producer is not accessing the buffer.
            After accessing completing the read on the buffer and decrementing the size variable, the consumer releases the mutex, and then signals that there is one more space available in the buffer via incrementing the \texttt{sem\_empty} semaphore.

            The producer waits on the \texttt{sem\_empty} semaphore to ensure that there is space left in the buffer; 
            that is, that the empty spaces semaphore has not reduced to zero.
            Like the consumer, the producer locks the mutex before adding to the buffer, then releases the lock before signaling that there is now one more item in the buffer by incrementing the \texttt{sem\_full} semaphore.\par
            
            We decided to implement the buffer as a stack (last-in-first-out) because it was the most straightforward implementation and no items have priority over others. 
            This allows our consumers to simply take from the '--buffer.size' slot of the buffer, and the producers to add to the 'buffer.size++' slot.
            The buffer is also a shared variable.\par
            
            Producers and consumers are simply pthreads that are initialized to run the produce and consume functions, respectively. 
            The number of producers and consumers are set in global constants, and each will produce/consume indefinitely. \par
            
            We tested with more consumers than producers to ensure that consumers would wait for items to be placed in the buffer, and similarly tested with more producers than consumers to ensure that producers would wait if the buffer is full. 
            While testing, we reduced the buffer size to 3 to make the output more readable and manageable.
            Reducing the buffer size and printing clear and descriptive output was helpful in ensuring our solution is correct. 
            Though the buffer is larger, we left the print statements, which print which producer produces which item and which consumer consumes which item so that the user can ensure the results make sense. 
            Additionally, each producer prints the buffer size and a visualization of the buffer after each addition to the buffer. 
            This approach especially helped to see that producers were waiting when the buffer was full, and to further ensure that reads and writes are not conflicting. 
            For example, if item a is produced, we should definitely see 'a' in the buffer immediately after, and a consumer should later consume 'a'.
            \par

            For the random number generators, we tested the Mersenne Twister on the os2 server, and we tested the rdrand assembly instruction on Brian's laptop in Bash on Ubuntu for Windows. 
            We used print statements to ensure that the random number generator took the expected branch of the random number generator. \par
          
            Before this assignment, we had both worked with mutexes, but had only a little knowledge on what semaphores were and how to use them. 
            We now know that mutexes are binary -- they lock when one thread accesses the critical code and unlock when it is done -- and that semaphores can allow a set number of threads to run before blocking. 
            Further, we have more of an understanding of what a semaphore is: essentially an integer that decrements when a thread hits it and increments when a thread leaves the code enclosed in the semaphore.
            We also learned that this is a fundamental aspect of how event-driven programs must work, in order to synchronize multiple threads and ensure that collisions do not occur.
            \par
    \end{singlespace}

\section{Version Control Log}
    \begin{tabular}{l l l}\textbf{Detail} & \textbf{Author} & \textbf{Description}\\\hline
\href{https://github.com/jlioy/group11repo/commit/32a12311779e0343d3747af7206d5b09710f27de}{32a1231} & Joshua Lioy & Changed to clean state.\\\hline
\href{https://github.com/jlioy/group11repo/commit/7c92e855b7cb500e9d4d589a63d0be44badad0ed}{7c92e85} & Joshua Lioy & Initial changes made to the config file.\\\hline
\href{https://github.com/jlioy/group11repo/commit/95518cf7b12613b9ae26d50e62ca1b13a76e8041}{95518cf} & Joshua Lioy & Made more changes to the .config file\\\hline
\href{https://github.com/jlioy/group11repo/commit/176f9e1edac99d8a29d5bc62acdb0ba0e8898f74}{176f9e1} & Joshua Lioy & Added the tex document and design plan.\\\hline
\href{https://github.com/jlioy/group11repo/commit/e5b27517b6cdc7151dd51476d7810beca9293f8c}{e5b2751} & Joshua Lioy & Initial implementation of the best-fit SLOB\\\hline
\href{https://github.com/jlioy/group11repo/commit/2aa44a87bf528370134d833a99afc6f50232563c}{2aa44a8} & Joshua Lioy & Made changes to syscalls files.\\\hline
\href{https://github.com/jlioy/group11repo/commit/6fbaea6c103f5103f3349679aea9c6a544862546}{6fbaea6} & Joshua Lioy & Made more changes to slob.c to fix compile errors\\\hline
\href{https://github.com/jlioy/group11repo/commit/f2c369b72df97f0ac983eb42062137a7ff619e87}{f2c369b} & Joshua Lioy & Added slob\_test.c\\\hline
\href{https://github.com/jlioy/group11repo/commit/113de27ff1d29a712c59522538bc7ddd60dc2200}{113de27} & Joshua Lioy & Final working version of bf slob.c\\\hline\end{tabular}

    
\section{Work Log}
    \textbf{Monday 10/2}
        \begin{itemize}
            \item 10:30am - 12:30pm, Brian and Joshua
            \begin{itemize}
                \item Familiarized ourselves with requirements
                \item Created source repo on GitHub
                \item Tried to run qemu, ran into problems
            \end{itemize}
            \item 1:30pm - 2:00pm, Brian
                \begin{itemize}
                    \item Office hours to figure out qemu-system command was wrong on website
                \end{itemize}
        \end{itemize}
    \textbf{Tuesday 10/3}
        \begin{itemize}
            \item 10:00am - 11:30am, Brian and Joshua:
            \begin{itemize}
                \item Figured out how to run qemu with GDB
                \item Copied config file, built new kernel instance
                \item Started figuring out how to create PDF with Latex
            \end{itemize}
        \end{itemize}
    \textbf{Wednesday 10/4}
        \begin{itemize}
            \item 10:00am - 12:00pm, Brian and Joshua
            \begin{itemize}
                \item Worked together on figuring out LaTeX
            \end{itemize}
            \item 1:00pm - 1:50pm, Brian
                \begin{itemize}
                    \item Attended office hours to ask about LaTeX issue and how to proceed on Assignment 1
                \end{itemize}
        \end{itemize}
    \textbf{Thursday 10/5}
        \begin{itemize}
            \item 10:00am - 10:45am, Brian and Joshua
            \begin{itemize}
                \item Completed the Kernel building portion of assignment 1
            \end{itemize}
        \end{itemize}
    \textbf{Friday 10/6}
        \begin{itemize}
            \item 9:00am - 11:00am, Brian
            \begin{itemize}
                \item Attended office hours while working on Concurrency assignment
            \end{itemize}
            \item 3:00pm – 6:00pm, Joshua
                \begin{itemize}
                    \item Worked on setting up LaTeX template and resolving issues with compiling on the OS2 server.
                \end{itemize}
        \end{itemize}
    \textbf{Saturday 10/7}
        \begin{itemize}
            \item 9:30am - 11:30am, Brian
            \begin{itemize}
                \item Worked on Concurrency portion of assignment 
            \end{itemize}
            \item 9:30am - 12:30pm, Joshua
                \begin{itemize}
                    \item Continued setting up LaTex template and fixing some issues with the repository.
                    \item Worked on the solution to the concurrency problem.
                \end{itemize}
            \item 12:30pm -2:00pm, Joshua
                \begin{itemize}
                    \item Got the repo fixed and started working on the write-up.
                \end{itemize}
            \item 8:00pm - 1:30am, Brian
                \begin{itemize}
                    \item Got the concurrency assignment working on os2 server (still needs rdrand work)
                    \item Tested it, appears to work when using Mersenne Twister
                    \item Emailed Joshua status update, plans for tomorrow
                \end{itemize}
        \end{itemize}
    \textbf{Sunday 10/9}
        \begin{itemize}
            \item 10:00am - 12:00pm, Brian
            \begin{itemize}
                \item Emailed Kevin about proper consumer behavior
                \item Made changes per Kevin's response
                \item Implemented extra credit
                \item Began working on concurrency write-up
            \end{itemize}
            \item 7:45pm - 10:30pm, Brian Joshua
                \begin{itemize}
                    \item Cleaned up concurrency solution, added more print statements
                    \item Changed seed for Mersenne twister to be command line arg
                    \item Worked on concurrency write up
                    \item Added to LaTeX makefile so it would build concurrency solution
                    \item Grouped all files together in shared folder
                    \item Implemented the rdrand portion of the concurrency problem
                \end{itemize}
        \end{itemize}
    \textbf{Monday 10/9}
        \begin{itemize}
            \item 8:00am - 9:30am Brian
            \begin{itemize}
                \item Figured out how to use BibTex
                \item Cited reference to Little Book of Semaphores in project write-up
            \end{itemize}
            \item 10:30am - 12:00am, Joshua
                \begin{itemize}
                    \item Added git log and work log to report.
                \end{itemize}
        \end{itemize}

\bibliography{project1cite}{}
\bibliographystyle{ieeetr}

\end{document}
