\documentclass[letterpaper, onecolumn, draftclsnofoot, 10pt, compsoc]{IEEEtran}

\usepackage{graphicx}                                        
\usepackage{amssymb}                                         
\usepackage{amsmath}                                         
\usepackage{amsthm}                                          
\usepackage{alltt}                                           
\usepackage{float}
\usepackage{color}
\usepackage{url}
\usepackage{balance}
\usepackage{enumitem}
\usepackage{pstricks, pst-node}
\usepackage{setspace}

\usepackage{geometry}
\geometry{textheight=9.5in, textwidth=6in}

\usepackage{listings}

\definecolor{codegreen}{rgb}{0,0.6,0}
\definecolor{codegray}{rgb}{0.5,0.5,0.5}
\definecolor{codepurple}{rgb}{0.58,0,0.82}
\definecolor{backcolour}{rgb}{0.95,0.95,0.92}

\lstdefinestyle{mystyle}{
  backgroundcolor=\color{backcolour},   commentstyle=\color{codegreen},
  keywordstyle=\color{magenta},
  numberstyle=\tiny\color{codegray},
  stringstyle=\color{codepurple},
  basicstyle=\footnotesize,
  breakatwhitespace=false,         
  breaklines=true,                 
  captionpos=b,                    
  keepspaces=true,                 
  numbers=left,                    
  numbersep=5pt,                  
  showspaces=false,                
  showstringspaces=false,
  showtabs=false,                  
  tabsize=2
}

\lstset{style=mystyle}

\newcommand{\cred}[1]{{\color{red}#1}}
\newcommand{\cblue}[1]{{\color{blue}#1}}

\usepackage{hyperref}
\hypersetup{
    colorlinks=false,
    pdfborder={0 0 0},
}
\usepackage{geometry}

\def \name{Joshua Lioy}
\def \GroupNumber{11}
\def \GroupMemberOne{Joshua Lioy}
\def \GroupMemberTwo{Brian Wiltse}

\begin{document}
\begin{titlepage}
    \pagenumbering{gobble}
    \begin{singlespace}
    	\includegraphics[height=4cm]{coe_v_spot1}
        \hfill 
        \par\vspace{.2in}
        \centering
        \scshape{
            \huge Operating Systems 2 \par
            {\large\today}\par
            \vspace{.5in}
            \textbf{\Huge Project 1: Getting Acquainted}\par
            \vfill
            \vspace{5pt}
            {\large Prepared by }\par
            Group \GroupNumber\par
            \vspace{5pt}
            {\Large
                \GroupMemberOne\par
                \GroupMemberTwo\par
            }
            \vspace{20pt}
        }
        \begin{abstract}
        This document covers Group 11's findings for assignment 1. This includes the commands used for the initial kernel setup as well as a write-up of the concurrency solution.
        \end{abstract}     
    \end{singlespace}
\end{titlepage}

\newpage
\pagenumbering{arabic}
\tableofcontents

\newpage
\section{Log of Commands Used to Set Up and Run the Virtual Linux Kernel}

    \subsection{Setting Up the Environment}
    \begin{lstlisting}[language=bash]
    source /scratch/fall2017/11/common/environment-setup-i586-poky-linux 
    \end{lstlisting}
    
    \subsection{Checking Out v3.19.2}
    \begin{lstlisting}[language=bash]
    git checkout "v3.19.2"
    \end{lstlisting}
    
    \subsection{Running qemu the First Time}
    \begin{lstlisting}[language=bash]
    qemu-system-i386 -gdb tcp::5511 -S -nographic -kernel bzImage-qemux86.bin -drive file=core-image-lsb-sdk-qemux86.ext4,if=virtio -enable-kvm -net none -usb -localtime --no-reboot --append "root=/dev/vda rw console=ttyS0 debug"
    \end{lstlisting}
    
    \subsection{Running qemu}
    \begin{lstlisting}[language=bash, caption=Original Kernel]
    qemu-system-i386 -gdb tcp::5511 -S -nographic -kernel bzImage-qemux86.bin -drive file=core-image-lsb-sdk-qemux86.ext4,if=virtio -enable-kvm -net none -usb -localtime --no-reboot --append "root=/dev/vda rw console=ttyS0 debug"
    \end{lstlisting}
    \begin{lstlisting}[language=bash, caption=Our Kernel]
    qemu-system-i386 -gdb tcp::5511 -S -nographic -kernel CS-444-Group-11/arch/x86/boot/bzImage -drive file=common/core-image-lsb-sdk-qemux86.ext4,if=virtio -enable-kvm -net none -usb -localtime --no-reboot --append "root=/dev/vda rw console=ttyS0 debug"
    \end{lstlisting}
    
    \subsection{Connecting to the Kernel Remotely Through GDB}
    \begin{lstlisting}[language=bash]
    target remote :5511
    continue
    \end{lstlisting}
    
    \subsection{Shutting Down the Virtual Machine}
    \begin{lstlisting}[language=bash]
    shutdown -h now 
    \end{lstlisting}
    
    \subsection{Building a New Instance of the Kernel}
    \begin{lstlisting}[language=bash]
    make -j4 all
    \end{lstlisting}

\newpage
\section{An Explanation of the Flags in the qemu Command-line}
    \begin{singlespace}
        \begin{itemize}
            \item -gdb: This flag provides a gdb stub which allows connection from any machine on the network. In our case this is another connection through the OS2 server.
            \item -S: This flag enforces manual CPU start. In our case the CPU will start after entering continue into gdb.
            \item -nographic: This flag disables graphics.
            \item -kernel: This flag specifies the location of the kernel file. Note that multiboot is not supported.
            \item -drive: This flag specifies the location of the virtual drive file as well as options for things such as the media type, drive file location, interface etc.
            \item -enable-kvm: This flag forces the use of KVM Virtualization.
            \item -net: This flag configures the network parameters.
            \item -usb: This flag enables the use of usb devices.
            \item -localtime: This flag sets the virtual machine to use the time of the local machine that it is running on. In our case this is the OS2 server time.
            \item --no-reboot: This flag prevents rebooting of the virtual machine.
            \item --append: This flag appends a command line parameter.
        \end{itemize}
    \end{singlespace}

\section{Concurrency Solution Write-Up}
    \begin{singlespace}
    %Introduction Here
    \par
        \begin{enumerate}
            \item \textit{What do you think the main point of this assignment is?}\par
            \par
            \item \textit{How did you personally approach the problem? Design decisions, algorithm, etc.}\par
            \par
            \item \textit{How did you ensure your solution was correct? Testing details, for instance.}\par
            \par
            \item \textit{What did you learn?}\par
            \par
        \end{enumerate}
    \end{singlespace}

\section{Version Control Log}

\section{Work Log}

%\lstinputlisting[language=C]{Q0.c}
\end{document}
